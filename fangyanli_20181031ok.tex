 \documentclass[14pt, a4paper,openany]{ctexbook} %CTEX报告文章格式
%\documentclass[14pt, a4paper,openright]{ctexbook} %CTEX报告文章格式   % openright  新章节仅从奇数页开始
%\documentclass[openany]{book} openany表示新章节可从奇数或偶数页开始
\usepackage{url}
\usepackage[top=2.5cm, bottom=3cm, left=3.8cm, right=2.5cm]{geometry}
\usepackage{xeCJK}
\setCJKmainfont[AutoFakeSlant=0.3]{KaiTi}
\setmainfont{Times New Roman}
\usepackage{fancyhdr}
\usepackage{lastpage}
\usepackage{layout}
\usepackage{geometry}
\usepackage{makecell}
\usepackage{graphicx}
\usepackage{amsmath,amssymb}
\usepackage{epstopdf}
\usepackage{endnotes}
\usepackage{indentfirst}
\usepackage{listings}
\usepackage{fancybox}
\usepackage[table]{xcolor}% http://ctan.org/pkg/xcolor
\usepackage{soul}
\usepackage[figuresright]{rotating}
\usepackage{lscape}
\usepackage{pdflscape}
\usepackage{verbatim}
\usepackage{setspace}
\usepackage{rotating} 
\usepackage{array, tabularx, caption, multirow,boldline,booktabs}
%\usepackage{subfigure}
%\usepackage{times}
%\usepackage{txfonts}
%\usepackage{fontspec}
\usepackage{adjustbox}
\usepackage{tikz}
\usepackage{longtable}

%\usepackage{xunicode}

%\usepackage{xltxtra}
\newcommand{\sihao}{\fontsize{14pt}{\baselineskip}\selectfont}% 四号
\newcommand{\yihao}{\fontsize{28pt}{\baselineskip}\selectfont} %一号

\newcommand{\erhao}{\fontsize{21pt}{\baselineskip}\selectfont} %二号

\newcommand{\xiaoerhao}{\fontsize{18pt}{\baselineskip}\selectfont} %小二号

\newcommand{\sanhao}{\fontsize{15.75pt}{\baselineskip}\selectfont} %三号

\renewcommand\thechapter{第\chinese{chapter}章}  %调整目录中的1 为中文下的第一章
\renewcommand\thesection{第\chinese{section}節}  %调整目录中的1.1为中文下的第一節



\makeatletter    %调整文中公式的變換
\@addtoreset{equation}{chapter}  %调整文中公式的變換
\makeatother  %调整文中公式的變換
\renewcommand\theequation{\arabic{chapter}.\arabic{equation}}  %调整文中公式的變換

\usepackage[titles]{tocloft}
\renewcommand{\cftdot}{$\cdot$}
\renewcommand{\cftdotsep}{1.5}
\setlength{\cftbeforechapskip}{0pt}
\renewcommand{\cftchapfont}{\zihao{4}}
\renewcommand{\cftchappagefont}{\zihao{4}}
\renewcommand{\cftchapleader}{\cftdotfill{\cftchapdotsep}}
\renewcommand{\cftchapdotsep}{\cftdotsep}
\makeatletter
\renewcommand{\numberline}[1]{%
\settowidth\@tempdimb{#1\hspace{0.5em}}%
\ifdim\@tempdima<\@tempdimb%
  \@tempdima=\@tempdimb%
\fi%
\hb@xt@\@tempdima{\@cftbsnum #1\@cftasnum\hfil}\@cftasnumb}
\makeatother

\allowdisplaybreaks[4]
\lstset{breaklines}
%\bibliographystyle{unsrt}

%\rhead{\fontsize{10pt}{\baselineskip}\selectfont\chaptermark}\rhead{\fontsize{10pt}{\baselineskip}\selectfont\leftmark}


\makeatletter   %调整参考文献的标号,让序号不要显示
\let\@@scshape=\scshape
\renewcommand{\scshape}{\ifnum\strcmp{\f@series}{bx}=\z@
\usefont{T1}{cmr}{bx}{sc}
\else\ifnum\strcmp{\f@shape}{it}=\z@
\fontshape{scsl}\selectfont
\else\@@scshape\fi\fi}
\renewcommand\@biblabel[1]{}
\makeatother  %调整参考文献的标号,让序号不要显示


% \usepackage{remreset}
% \makeatletter
% \@removefromreset{table}{chapter}
% \@removefromreset{figure}{chapter}
% \makeatother


\usepackage{caption}


\usepackage{titlesec}

%\titleclass{\chapter}{straight}  %设置标题
%\titleformat{\chapter}[hang]{\centering\fontsize{18pt}{0}\selectfont}{{\thechapter}}{1em}{}

%\titlespacing{\chapter}{0pt}{-20pt}{25pt}

\renewcommand{\chaptername}{{\thechapter}}
\titleformat{\chapter}{\centering \fontsize{22pt}{1pt}\selectfont}{第\chinese{chapter}章}{1em}{\thispagestyle{fancy}}[\vspace{-0.5cm}]
%\titleformat{\chapter}{\centering\sanhao\hei}{第\,\thechapter\,章}{1em}{}[\vspace{-1cm}]


\titleformat{\section}{\centering\fontsize{18pt}{1pt}\selectfont}{第\chinese{section} 節}{0.5em}{}

\titleformat{\subsection}{\fontsize{16pt}{\baselineskip}\selectfont}{\chinese{subsection}、}{0.5em}{}

\titleformat{\subsubsection}{\fontsize{14pt}{\baselineskip}\selectfont}{\thesubsubsection}{1em}{}

\titleformat{\paragraph}{\fontsize{14pt}{\baselineskip}\selectfont\bf}{\theparagraph}{1em}{}
\titlespacing{\chapter}{0pt}{-5pt}{45pt}
\titlespacing*{\section} {0pt}{3.25ex plus 1ex minus .2ex}{0.5ex plus .2ex}
\titlespacing*{\subsection}{0pt}{1.25ex plus .2ex minus .2ex}{0.5ex plus .2ex}
\titlespacing*{\subsubsection}{0pt}{1.25ex plus 1ex minus .2ex}{0.5ex plus .2ex}


%\titlecontents{chapter}[0pt]{\addvspace{2pt}\filright}
%              {\contentspush{ {\thecontentslabel}\  }}
%              {}{\titlerule*[8pt]{.}\contentspage}
%
%\titlecontents{section}[20pt]{\addvspace{2pt}\filright}
%              {\contentspush{\thecontentslabel\  }}
%              {}{\titlerule*[8pt]{.}\contentspage}
\makeatletter
\renewcommand{\numberline}[1]{%
\settowidth\@tempdimb{#1\hspace{0.5em}}%
\ifdim\@tempdima<\@tempdimb%
  \@tempdima=\@tempdimb%
\fi%
\hb@xt@\@tempdima{\@cftbsnum #1\@cftasnum\hfil}\@cftasnumb}
\makeatother
\linespread{1.3} %\linespread{1.3}产生1.5倍行距,1.6产生双倍行距,

\title{澳門科技大學\\
  研究生畢業論文}
\author{fan }
\date{\today}






\numberwithin{figure}{chapter}
\numberwithin{table}{chapter}
\renewcommand{\thetable}{\arabic{chapter}--\arabic{table}}
\renewcommand{\thefigure}{\arabic{chapter}--\arabic{figure}}
\begin{document}
\sihao\linespread{1.1} \parskip=0pt
\begin{titlepage}




\thispagestyle{empty}
\vspace*{3cm}
 \begin{center}


\newcommand{\PreserveBackslash}[1]{\let\temp=\\#1\let\\=\temp}
\zihao{2}
\begin{tabular}{cp{4in}}  %控制表格寬度,自動換行
題目: & \PreserveBackslash\raggedright  %控制表格寬度,自動換行
   中文标题\\
\end{tabular}

\vspace{1cm}

\zihao{-2}
%\Roman{
\begin{tabular}{cp{4in}}
Title: & \PreserveBackslash\raggedright
   title \\
\end{tabular}


 \vspace{1cm}

\end{center}
\newcommand{\unline}[2][6cm]{\underline{\hbox to #1{\hfill #2 \hfill}}}

\begin{center}
{\fontsize{12pt}{\baselineskip}\selectfont
 \vspace{2cm}
\begin{center}
\begin{table}[htbp]
  \centering
    \begin{tabular}{llr}
    	\makecell*[l{p{8em}}]{ \sanhao 姓\ \ ~~~~~名\\ \sanhao Name}:    & \unline{\sanhao  } \\
    	\makecell*[l{p{8em}}]{ \sanhao 學\ \ ~~~~~號\\ \sanhao Student No.}: & \unline{\sanhao  } \\
    	\makecell*[l{p{8em}}]{ \sanhao 學\ \ ~~~~~院\\ \sanhao Faculty}: & \unline{\sanhao 商學院} \\
    	\makecell*[l{p{8em}}]{ \sanhao 課\ \ ~~~~~程\\ \sanhao Program}: & \unline{\sanhao 管理學博士} \\
    	\makecell*[l{p{8em}}]{ \sanhao 專\ \ ~~~~~業\\ \sanhao Major}: & \unline{\sanhao  } \\
    	\makecell*[l{p{8em}}]{ \sanhao 指導老師\\ \sanhao Supervisor}: & \unline{\sanhao   } \\
    	\makecell*[l{p{8em}}]{ \sanhao 日\ \ ~~~~~期\\ \sanhao Date}: & \unline{\sanhao  } \\
    \end{tabular}%

\end{table}%
\end{center}



}
\end{center}

\end{titlepage}
\sihao\linespread{1.3} \parskip=0pt
\pagestyle{fancy}                    % 设置页眉
\lhead{論文題目}
%\chead{页眉中间}
%\renewcommand{\chaptername}{{\thechapter}}
\rhead{\leftmark}
\pagenumbering{Roman}%定义页码显示形式
\pagestyle{fancy} %定义此頁不顯示頁眉,但是頁腳中部顯示頁碼,plain替換為empty,則頁眉頁腳均不顯示

\chapter*{摘~~~~要}
\markboth{摘~~~~要}{}
%\renewcommand{\chaptername}{摘要} %此行為定義頁眉右邊顯示內容,但是208定義了此頁無頁眉,因此無用
\addcontentsline{toc}{chapter}{摘要}  %上一行在chapter后面加*,即让这章无编号,也不在目录中显示,加上这行后,在目录中显示
 %調整摘要頁行距與字號
摘要內容.....









\vspace{1.6cm}
 \bigskip
\noindent \textbf{關鍵詞}關鍵字1;關鍵2。


%\英文摘要部分 abstract


\newpage


\thispagestyle{plain} %定义此頁不顯示頁眉,但是頁腳中部顯示頁碼,plain替換為empty,則頁眉頁腳均不顯示
\sihao
\chapter*{Abstract}
\markboth{Abstract}{}
%\renewcommand{\chaptername}{摘要} %此行為定義頁眉右邊顯示內容,但是208定義了此頁無頁眉,因此無用
\addcontentsline{toc}{chapter}{Abstract}  %上一行在chapter后面加*,即让这章无编号,也不在目录中显示,加上这行后,在目录中显示

\begin{spacing}{1.3}{\sihao
\tolerance=1
\emergencystretch=\maxdimen
\hyphenpenalty=10000
\hbadness=10000



\vspace{1.5cm}
\noindent \textbf{Keywords:} words, words .
 }
\end{spacing}
\newpage



\thispagestyle{plain}
\renewcommand{\contentsname}{目~~~~錄}
\renewcommand{\chaptername}{目錄}
\addcontentsline{toc}{chapter}{目錄}
\setcounter{tocdepth}{1}
\begin{spacing}{1.35}\sihao
\tableofcontents
\end{spacing}
\newpage




\pagenumbering{arabic}
\chapter{緒論}
\renewcommand{\chaptername}{第\chinese{chapter}章~~緒論}%頁眉右側顯示本章標題
%\linespread{1.1}\selectfont
\linespread{1.3}\selectfont
\section{研究背景及意義}
背景意義



\section{研究目的}





\section{研究內容}







\section{研究思路與方法}



\section{研究創新點}




\chapter{相關研究綜述}
\renewcommand{\chaptername}{第\chinese{chapter}章~~相關研究綜述}
涉及的研究問題綜述



\section{問題1}



\section{問題2}







\section{管理}



\subsection{管理1}





\subsection{管理2}





\section{文獻評述}
分析當前研究現狀,本研究的不同點


\newpage
\chapter{基礎模型}
\renewcommand{\chaptername}{第\chinese{chapter}章~~基礎模型}
\section{問題描述}


\section{模型與參數}





\section{研究問題1}

\subsection{決策1}




\subsection{決策2}




\subsubsection{(一)三級標題1}

\subsubsection{(二)三級標題2}

\section{研究問題2}

\subsection{決策3}







\subsection{決策4}



\section{算例分析}



\subsection{分析1}


\subsection{分析2}

敏感性分析

\vspace{-0.4cm}
\subsubsection{(三)三級標題}



\subsubsection{(四)三級標題}



\section{本章小結}



%\newpage 我们在使用Latex时,有时需要另起一页继续写。这时,你最好使用“\clearpage” 实现这个功能,而不要用“\newpage”。
\clearpage


\chapter{擴展模型一}
\renewcommand{\chaptername}{第\chinese{chapter}章~~擴展模型一}
\section{問題描述}

描述研究的問題。。。。

\section{二級標題}

\subsection{決策1}
  研究對象的決策
\subsection{決策2}

\section{算例分析}
\subsection{三級標題}



\section{本章小結}




\newpage
\chapter{擴展模型二}
\renewcommand{\chaptername}{第\chinese{chapter}章~~擴展模型二}

\section{問題描述}

\section{模型與參數}



\subsection{對象1}


\subsection{對象2}

\section{本章小結}



\newpage
\chapter{結論與展望}
\renewcommand{\chaptername}{第\chinese{chapter}章~~結論與展望}
\section{結論}



\section{展望}







\newpage
%\setCJKmainfont[BoldFont = STKaiti, ItalicFont = STKaiti]{STKaiti}設置主體字體與斜體字體,意大利格式即為斜體字體
\renewcommand\bibname{參考文獻}
\begin{thebibliography}{999}
\renewcommand{\chaptername}{~~~參考文獻}
\markboth{參考文獻}{參考文獻}
\addcontentsline{toc}{chapter}{参考文獻}




\end{thebibliography}









\newpage
\chapter*{致謝}
\markboth{致謝}{致謝}
\addcontentsline{toc}{chapter}{致謝}
\renewcommand{\chaptername}{第\chinese{chapter}章~~致謝}

 







\vspace{2cm}
\rightline{方~~艷~~麗}


\rightline{澳門科技大學商學院}


\rightline{二零一八年十月}

\newpage
\chapter*{個人簡歷}
\markboth{個人簡歷}{個人簡歷}
\addcontentsline{toc}{chapter}{個人簡歷}
\renewcommand{\chaptername}{個人簡歷}


\end{document}
